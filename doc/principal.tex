\documentclass[a4paper, onecolumn, 12pt, final]{article}
%\documentclass[a4paper, onecolumn, 12pt, draft]{article}
%\documentclass[a4paper, twocolumn, 10pt, draft]{article}

\usepackage[utf8]{inputenc}
\usepackage[T1]{fontenc}

% aptitude install texlive-lang-spanish
\usepackage[catalan]{babel}

\usepackage[pdftex]{color,graphicx}
\usepackage[font=small,format=plain,labelfont=bf,up,textfont=it,up]{caption}
\usepackage{url}
\usepackage[plainpages=false,pdfpagelabels,bookmarks=true]{hyperref}
\usepackage{textcomp}
\usepackage{sverb}
\usepackage{listings}
\usepackage{verbatim}
\usepackage[kerning,spacing]{microtype}
\microtypecontext{spacing=nonfrench}

% aptitude install texlive-fonts-extra
\usepackage{newcent}

\pagestyle{headings}

\title{EmoFS: sistema de fitxers amb inodes com a pràctica per Ampliació de
  Sistemes Operatius (4520)}

\author{Bartomeu Miró Mateu <bartomeumiro@gmail.com>, \\
  DNI 43182841-L, Informàtica Tècnica de Sistemes (TIS2)\\
        Pau Ru\l.lan Ferragut <paurullan@gmail.com>, \\
  DNI 43142697-X, Informàtica Tècnica de Sistemes (TIS2) }

\date{15 d'agost de 2009}

\pagestyle{headings}

\begin{document}

\maketitle

\begin{abstract}
Un sistema de fitxers és el component del sistema operatiu que facilita als
usuaris desar informació i ordenar-ho per fitxers i directoris. Aquest document
és la memòria d'un sistema de fitxers basat en inodes amb la finalitat
d'experimentar el que s'ha estudiat a l'assignatura d'ampliació de sistemes
operatius (codi 4520).
\end{abstract}



\section{Descripció del problema}

El temari de l'assignatura d'ampliació de sistemes operatius és principalment
l'estudi de la memòria virtual i els sistemes de fitxers. Un sistema de fitxers
és la part del sistema operatiu encarregat de mantenir permanentment les dades
de l'usuari i els programes del sistema. Com a pràctica de l'assignatura
s'implementa un petit sistema de fitxers basat en inodes i amb capacitat de
directoris en arbre. Amb el llenguatge de programació C haurem d'escriure no
sols les eines típiques del UNIX com crear un directori o esborrar un fitxer
sinó també un simulador que executi una càrrega de treball contra el nostre
sistema.

Tal i per seguir les tradicions del UNIX vàrem decidir anomenar el nostre
sistema de fitxers (\emph{SF} a partir d'ara) algunacosa-fs. Com que un dels
membres del grup estava especialment trist en l'època de començar la pràctica
usarem \emph{emo} dels \emph{emotive hardcore}, gent que segons la cultura
popular sempre està deprimida i en conseqüència anomenam al nostre SF
\emph{emofs}.

L'estructura general de la pràctica ve indicada per l'enunciat exposat pels
professors: nou setmanes de feina on seqüencialment es van introduint distintes
capes de la feina. Des de la primera setmana on cream una imatge i escrivim
blocs d'informació fins a la darrera on treballam amb fitxers dins la nostra
imatge de sistema de fitxers.


\section{Disseny i estructura del SF}

La pràctica en sí està dividida en quatre seccions: les biblioteques, les eines
d'interacció, els jocs de proves i el simulador. Anomenam biblioteca a tot el
codi que l'usuari final no usa directament. Aquesta secció és el major cos de
la pràctica en nombre de línies perquè és la implementació del sistema de
fitxers pròpiament dit.

Les altres tres seccions formen el suport per l'ús del sistema de
fitxers. Anomenam a les eines d'interacció a tots els petits programes clàssics
del UNIX que s'usen al dia a dia: \verb+cd+, \verb+cat+, \verb+ls+ o
\verb+mkdir+. El simulador és la peça indicada a l'enunciat que fa la prova del
sistema. El simulador està completament definit i no és més que crear un gran
nombre de clients que facin un directori i duguin a terme escriptures a un
fitxer. Els jocs de proves són petits programes que tenen com a únic objectiu
comprovar la correcció de cadascun dels mòduls.

L'estructura de les biblioteques és un conjunt de vàries capes més un fitxer
comú a totes elles. L'estructura completa es pot estudiar a la taula
\ref{capes:simple} però bàsicament està formada pel nucli del sistema de
fitxers, les llibreries d'accés al nucli, la gestió de fitxers i les eines
d'usuari. Aquest element al que totes tenen accés s'anomena \verb+common.h+ i
allà s'emmagatzemen variables globals de configuració com el grandària de bloc
predeterminat. Les dades i valors que apareixen al fitxer comú poden
modificar-se a la creació de la imatge del SF.

\begin{table}
\centering
\begin{tabular}{|ll|}
\textbf{Nivell} & \textbf{Components} \\
\hline
simulació   & sim.c \\
\hline
eines & mi\_mkfs.c \\
      & mi\_rm.c \\
      & mi\_ln.c \\
      & $\cdots$ \\
\hline
fitxers & dir \\
        & file \\
        & inode \\
\hline
llibreries & block\_lib \\
\hline
nucli & block \\
      & super \\
      & bitmap \\
\hline
\end{tabular}
\caption{Esquema dels mòduls del SF}
\label{capes:simple}
\end{table}

\begin{table}
\centering
\begin{tabular}{|l|l|l|l|}
\hline
\multicolumn{3}{|c|}{Metadades} & Dades \\
\hline
Superbloc & Mapa de bits & Inodes & Dades \\
\hline
\end{tabular}
\caption{Esquema de l'ordenació de les dades del SF}
\label{esquema}
\end{table}


La informació permanent del sistema de fitxers està separada en dues parts: les
\emph{dades} i les seves \emph{metadades}. Les metadades estan compostes pel
\emph{superbloc}, el \emph{mapa de bits} i pel vector d'\emph{inodes}.  Aquest
esquema es pot estudiar a la taula \ref{esquema}. El superbloc s'encarrega de
mantenir informació com quants de blocs queden lliures o quin és el primer
inode i el mapa de bits té com a funció indicar quins blocs de dades estan
lliures i quins estan ocupats.

El vector d'inodes és un conjunt d'estructures que assenyalen les
característiques d'un fitxer o directori: el seu tamany, la darrera
modificació, quants d'enllaços té i on es poden trobar els blocs físics on es
desa la informació.  Un inode es pot comportar com un fitxer de dades, on es
desa informació, o com una entrada de directori, un text que sosté l'estructura
d'arbre dels fitxers.


\section{Implementació}

Per simplicitat la gestió de concurrència es fa mitjançant un sol \emph{mutex}
que bloca totes les estructures. Aquest \emph{biglock} implica un rendiment
molt baix doncs sols permet una operació alhora però redueix enormement la
complexitat i els possibles problemes d'accés simultani. Tots els controls es
troben dins les funcions de \verb+dir.h+. També creàrem unes eines de
\verb+mi_mount+ i \verb+mi_umount+ que s'encarreguen de crear i destruir els
semàfors. Aquestes eines s'han d'executar abans de començar la interacció amb
el sistema de fitxers però no cal fer-ho pel simulador perquè el propi
\verb+sim.c+ ja ho fa.

Les variables i característiques com els tamanys de bloc, el nombre d'entrades
d'inode o el nombre de blocs de dades estan definits als fitxers corresponents:
\verb+common.h+, \verb+super.h+, \verb+inode.h+ \ldots Podem trobar que totes
les dades indirectes, com el tamany de \emph{padding} del superbloc o el nombre
final d'entrades al mapa de bits són calculades mitjançant macros però
recordant que s'ha recompilar el projecte si es vol fer algun canvi.

Com a funcions i tasques extres hem desenvolupat els programes d'usuari
\verb+mkdir+, \verb+touch+, \verb+append+. Amb aquests programes són molt útils
per fer comprovacions de com està la imatge del sistema de fitxers. Alhora així
podem escriure el simulador d'una manera especial: fent que cada tasca del fils
usi les eines d'usuari normals, imitant més el comportament d'usuari.

El \verb+ls+ accepta varis camins com entrada. És l'únic programa que ho fa
però sols ho hem volgut implementar com a prova conceptual: fer-ho per eines
com \verb+rm+ sols era un poc més de feina. Aquest processament es podria haver
fet amb una biblioteca d'arguments però l'hem fet a mà.

El programa \verb+mi_ls+ usa com a sortida un \verb+emofs_extract_path+ (el que
a l'enunciat és un \verb+info_fichero+) i aquestes funcions demanen un
\emph{buffer} per recórrer els fitxers d'un directori. Aquest \emph{buffer} pot
ser fixo, que malbarata l'espai, o dinàmic, amb el perill de quedar-se sense
memòria. Nosaltres hem usat la memòria dinàmica per així recordar els usos de
\verb+malloc+ i \verb+realloc+ sempre recordant que la pràctica no pretén tenir
directoris amb un gran nombre d'entrades.

\section{Jocs de proves i resultats}

Els jocs de proves no són eines d'usuari perquè no tenen cap utilitat més enllà
de fer les comprovacions de les feines setmana a setmana. Segons el que s'hagi
treballat durant la creació d'un joc de proves potser s'usen directament les
biblioteques del SF o alguna eina com el \verb+mkdir+. El codi d'aquestes
proves es pot veure a l'apèndix.

\subsection{Execucions}

Aquestes són les sortides de les execucions de les distintes eines d'usuari i
el simulador. Les execucions estan fetes per ordre sobre la mateixa imatge de
sistema de fitxers.


\begin{footnotesize}
\begin{verbatim}
mi_mkfs
mi_mount
mi_touch /x
mi_mkdir /a
mi_mkdir /a/b
mi_touch /a/b/y

mi_ls /
Type 	 Size 		 Epoc 		 Name 
d 	 64 		 1250506318 	 a 
f 	 0 		 1250506278 	 x 
d 	 0 		 1250506266 	 authors 

mi_ls /a
Size 		 Epoc 		 Name 
d 	 64 		 1250506372 	 b 

mi_write /a/b/y "prova de text molt llarg"

mi_cat /a/b/y
prova de text molt llarg

mi_write /a/b/y "MES TEXT QUE SOBREPASI" 5

mi_cat /a/b/y
provaMES TEXT QUE SOBREPASI

mi_ln /a/b/y /z

mi_cat /z
provaMES TEXT QUE SOBREPASI

mi_rm /a/b/y

mi_ls /a/b
Type 	 Size 		 Epoc 		 Name 

mi_umount
\end{verbatim}
\end{footnotesize}

\subsection{Simulador}

\begin{footnotesize}
\begin{verbatim}

Inici worker 13861
sim: worker: final client amb pid 13861
Inici worker 13862
sim: worker: final client amb pid 13862
Inici worker 13863
sim: worker: final client amb pid 13863
[...]
Inici worker 13959
sim: worker: final client amb pid 13959
Inici worker 13960
sim: worker: final client amb pid 13960
Type 	 Size 		 Epoc 		 Name 
d 	 64 		 1250518506 	 process_13960 
d 	 64 		 1250518506 	 process_13959 
d 	 64 		 1250518506 	 process_13958 
d 	 64 		 1250518506 	 process_13957 
d 	 64 		 1250518506 	 process_13956 
d 	 64 		 1250518506 	 process_13955 
d 	 64 		 1250518506 	 process_13954 
d 	 64 		 1250518506 	 process_13953 
d 	 64 		 1250518506 	 process_13952 
d 	 64 		 1250518506 	 process_13951 
d 	 64 		 1250518506 	 process_13950 
d 	 64 		 1250518506 	 process_13949 
d 	 64 		 1250518506 	 process_13948 
d 	 64 		 1250518506 	 process_13947 
d 	 64 		 1250518506 	 process_13946 
d 	 64 		 1250518506 	 process_13945 
d 	 64 		 1250518506 	 process_13944 
d 	 64 		 1250518506 	 process_13943 
d 	 64 		 1250518506 	 process_13942 
d 	 64 		 1250518506 	 process_13941 
d 	 64 		 1250518506 	 process_13940 
d 	 64 		 1250518506 	 process_13939 
d 	 64 		 1250518506 	 process_13938 
d 	 64 		 1250518506 	 process_13937 
d 	 64 		 1250518506 	 process_13936 
d 	 64 		 1250518506 	 process_13935 
d 	 64 		 1250518506 	 process_13934 
d 	 64 		 1250518506 	 process_13933 
d 	 64 		 1250518506 	 process_13932 
d 	 64 		 1250518506 	 process_13931 
d 	 64 		 1250518506 	 process_13930 
d 	 64 		 1250518506 	 process_13929 
d 	 64 		 1250518506 	 process_13928 
d 	 64 		 1250518506 	 process_13927 
d 	 64 		 1250518506 	 process_13926 
d 	 64 		 1250518506 	 process_13925 
d 	 64 		 1250518506 	 process_13924 
d 	 64 		 1250518506 	 process_13923 
d 	 64 		 1250518506 	 process_13922 
d 	 64 		 1250518506 	 process_13921 
d 	 64 		 1250518506 	 process_13920 
d 	 64 		 1250518506 	 process_13919 
d 	 64 		 1250518506 	 process_13918 
d 	 64 		 1250518506 	 process_13917 
d 	 64 		 1250518506 	 process_13916 
d 	 64 		 1250518506 	 process_13915 
d 	 64 		 1250518506 	 process_13914 
d 	 64 		 1250518506 	 process_13913 
d 	 64 		 1250518506 	 process_13912 
d 	 64 		 1250518506 	 process_13911 
d 	 64 		 1250518506 	 process_13910 
d 	 64 		 1250518506 	 process_13909 
d 	 64 		 1250518506 	 process_13908 
d 	 64 		 1250518506 	 process_13907 
d 	 64 		 1250518506 	 process_13906 
d 	 64 		 1250518506 	 process_13905 
d 	 64 		 1250518506 	 process_13904 
d 	 64 		 1250518506 	 process_13903 
d 	 64 		 1250518506 	 process_13902 
d 	 64 		 1250518506 	 process_13901 
d 	 64 		 1250518506 	 process_13900 
d 	 64 		 1250518506 	 process_13899 
d 	 64 		 1250518506 	 process_13898 
d 	 64 		 1250518506 	 process_13897 
d 	 64 		 1250518506 	 process_13896 
d 	 64 		 1250518506 	 process_13895 
d 	 64 		 1250518506 	 process_13894 
d 	 64 		 1250518506 	 process_13893 
d 	 64 		 1250518506 	 process_13892 
d 	 64 		 1250518506 	 process_13891 
d 	 64 		 1250518506 	 process_13890 
d 	 64 		 1250518506 	 process_13889 
d 	 64 		 1250518506 	 process_13888 
d 	 64 		 1250518506 	 process_13887 
d 	 64 		 1250518506 	 process_13886 
d 	 64 		 1250518506 	 process_13885 
d 	 64 		 1250518505 	 process_13884 
d 	 64 		 1250518505 	 process_13883 
d 	 64 		 1250518505 	 process_13882 
d 	 64 		 1250518505 	 process_13881 
d 	 64 		 1250518505 	 process_13880 
d 	 64 		 1250518505 	 process_13879 
d 	 64 		 1250518505 	 process_13878 
d 	 64 		 1250518505 	 process_13877 
d 	 64 		 1250518505 	 process_13876 
d 	 64 		 1250518505 	 process_13875 
d 	 64 		 1250518505 	 process_13874 
d 	 64 		 1250518505 	 process_13873 
d 	 64 		 1250518505 	 process_13872 
d 	 64 		 1250518505 	 process_13871 
d 	 64 		 1250518505 	 process_13870 
d 	 64 		 1250518505 	 process_13869 
d 	 64 		 1250518505 	 process_13868 
d 	 64 		 1250518505 	 process_13867 
d 	 64 		 1250518505 	 process_13866 
d 	 64 		 1250518505 	 process_13865 
d 	 64 		 1250518505 	 process_13864 
d 	 64 		 1250518505 	 process_13863 
d 	 64 		 1250518501 	 process_13862 
d 	 64 		 1250518501 	 process_13861 

sim: anam a mostrar un fitxer dels creats
/simul_200981716151/process_13861/prueba.dat
-------------------------------------
16:15:1 escriptura nombre 35 a la posicio 14
mbre 14 a la posicio 38
16:15:1 escriptu16:15:1 escriptura nombre 11 a la posicio 113
16:15:1 escriptura nombre 1 a la posici16:15:1 escriptura nombre 4 a la posicio 275
16:1516:15:1 escriptura nombre 49 a la posicio 356
o 362
16:15:1 escript16:15:1 escriptura nombre 43 16:15:1 escriptura nombre 47 a la posicio 487
16:15:1 escriptura nombre 42 a la posicio 556
6:15:1 escriptura nombre 27 a la posicio 601
osicio 612
la p16:15:1 escriptura nombre 15 a la posicio 662
16:15:1 escriptura nombre 12 a la posicio 755
16:15:1 escriptura nombre 48 a la posicio 809
16:15:1 escriptura n16:15:1 escriptura nombre 44 a la posicio 916:15:1 escriptura nombre 45 a la posicio 946
ra nombre 0 a la posicio 976
16:15:1 escriptura nombre 23 a la posicio 1034
16:15:1 escriptura nombre 46 a la posicio 1121
io 1129
 29 a la posicio 1151
sicio 1162
16:15:1 escriptura nombre 16 a la posicio 1232
16:15:1 escriptura nombre 38 a la posicio 1304
16:15:1 escriptura nombre 40 a la posicio 1351
riptura nombre 32 a la posicio 1387
16:15:1 escriptura nombre 19 a la posicio 1485
mbre 17 a la posicio 1511
 escrip16:15:1 escriptura nombre 21 a la posicio 1565
escriptura nombre 20 a la posicio 1604
16:15:1 escriptura nombre 34 a la posicio 1661
16:15:1 escriptura nombre 37 a la posicio 1758
o 1765
cio16:15:1 escriptura nombre 36 a la posicio 1815
cio 1824
16:15:1 escriptura nombre 33 a la posicio 1912
16:15:1 escriptura nombre 3 a la posicio 1970
-------------------------------------
Simulació acabada
\end{verbatim}
\end{footnotesize}

\section{Conclusions}

Hem escrit des de zero un sistema de fitxers d'usuari amb inodes i directoris
en arbre. Hem programat les eines mínimes per treballar sobre el nostre SF i
executat una completa simulació per observar el seu comportament. Com a
resultat tenim un sistema de fitxers que malgrat no l'usariem al nostre dia a
dia té les característiques d'un sistema de fitxers clàssic. És una prova
conceptual completa del funcionament d'una peça important dels nostres sistemes
operatius.

D'aquesta manera tancam una pràctica on queda constància de la necessitat d'un
bon disseny, d'unes bones interfícies entre capes i de les infinites
possibilitats i característiques que li podem afegir. Algunes coses com el
\verb+FUSE+ \footnote{ \url{http://fuse.sourceforge.net/} }, un sistema de
fitxers a nivell d'usuari, o les memòries cau plantejaven un repte molt
interessant però la seva complexitat sobrepassa les expectatives de la
pràctica. De tota manera queda clar podriem dedicar-li moltíssimes més hores,
fer grups de quatre o sis persones i encara faltaria temps per implementar
totes les parts que defineix el que és un sistema de fitxers modern.


\appendix
\section{Codi font}

% el codi font escrit

% http://www.cl.cam.ac.uk/~rf10/pstex/latexcommands.htm
% http://stackoverflow.com/questions/193298/best-practices-in-latex/897728#897728
% http://stackoverflow.com/questions/741985/latex-source-code-listing-like-in-professional-books/742012#742012
% http://stackoverflow.com/questions/1116266/listings-in-latex-with-utf-8-or-at-least-german-umlauts

\lstset{
  language=C,
  basicstyle=\ttfamily\small,
%  basicstyle=\ttfamily\footnotesize,
  columns=fullflexible,
  showstringspaces=false,
  numbers=left,
  numberstyle=\tiny,
  frame=tb,
  columns=fullflexible,
  morekeywords={assert},
  extendedchars=\true,
  inputencoding=utf8
}

\addtolength{\hoffset}{-1.5cm}

\subsection{Capa base}

%% bitmap.h
%% bitmap.c
%% block.h
%% block.c
%% block_lib.c
%% block_lib.h
%% common.h

\subsubsection{bitmap.h}
\lstinputlisting{src/bitmap.h}
\subsubsection{bitmap.c}
\lstinputlisting{src/bitmap.c}

\subsubsection{block.h}
\lstinputlisting{src/block.h}
\subsubsection{block.c}
\lstinputlisting{src/block.c}

\subsubsection{block\_lib.h}
\lstinputlisting{src/block_lib.h}
\subsubsection{block\_lib.c}
\lstinputlisting{src/block_lib.c}

\subsubsection{common.h}
\lstinputlisting{src/common.h}

\subsection{Nucli}

%% super.c
%% super.h
%% inode.c
%% inode.h
%% data_block.c
%% data_block.h
%% dir.c
%% dir.h
%% file.c
%% file.h

\subsubsection{super.h}
\lstinputlisting{src/super.h}
\subsubsection{super.c}
\lstinputlisting{src/super.c}

\subsubsection{inode.h}
\lstinputlisting{src/inode.h}
\subsubsection{inode.c}
\lstinputlisting{src/inode.c}

\subsubsection{file.h}
\lstinputlisting{src/file.h}
\subsubsection{file.c}
\lstinputlisting{src/file.c}

\subsubsection{dir.h}
\lstinputlisting{src/dir.h}
\subsubsection{dir.c}
\lstinputlisting{src/dir.c}

\subsection{Eines de sistema}

%% sem.c
%% sem.h

\subsubsection{sem.h}
\lstinputlisting{src/sem.h}
\subsubsection{sem.c}
\lstinputlisting{src/sem.c}


\subsection{Simulador}

%% sim.c

\subsubsection{sim.c}
\lstinputlisting{src/sim.c}

\subsection{Eines d'usuari}

%% mi_mkfs.c
%% mi_cat.c
%% mi_ln.c
%% mi_ls.c
%% mi_mkdir.c
%% mi_rm.c
%% mi_stat.c
%% mi_touch.c
%% mi_append.c
%% mi_write.c
%% mi_mount.c
%% mi_umount.c

\subsubsection{mi\_append.c}
\lstinputlisting{src/mi_append.c}
\subsubsection{mi\_mkfs.c}
\lstinputlisting{src/mi_mkfs.c}
\subsubsection{mi\_cat.c}
\lstinputlisting{src/mi_cat.c}
\subsubsection{mi\_ln.c}
\lstinputlisting{src/mi_ln.c}
\subsubsection{mi\_ls.c}
\lstinputlisting{src/mi_ls.c}
\subsubsection{mi\_mkdir.c}
\lstinputlisting{src/mi_mkdir.c}
\subsubsection{mi\_rm.c}
\lstinputlisting{src/mi_rm.c}
\subsubsection{mi\_stat.c}
\lstinputlisting{src/mi_stat.c}
\subsubsection{mi\_touch.c}
\lstinputlisting{src/mi_touch.c}
\subsubsection{mi\_write.c}
\lstinputlisting{src/mi_write.c}
\subsubsection{mi\_mount.c}
\lstinputlisting{src/mi_mount.c}
\subsubsection{mi\_umount.c}
\lstinputlisting{src/mi_umount.c}

\subsection{Jocs de proves}

%% tests_mapa_bits.c
%% tests_setmana1.c
%% tests_setmana2.c
%% tests_setmana3.c
%% tests_setmana5.c

\subsubsection{test\_mapa\_bits.c}
\lstinputlisting{src/tests_mapa_bits.c}

\subsubsection{test\_setmana1.c}
\lstinputlisting{src/tests_setmana1.c}

\subsubsection{test\_setmana2.c}
\lstinputlisting{src/tests_setmana2.c}

\subsubsection{test\_setmana3.c}
\lstinputlisting{src/tests_setmana3.c}

\subsubsection{test\_setmana5.c}
\lstinputlisting{src/tests_setmana5.c}


\addtolength{\hoffset}{1.5cm}



\newpage
\tableofcontents

\end{document}
