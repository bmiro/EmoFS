
\section{Disseny i estructura del SF}

La pràctica en sí està dividida en quatre seccions: les biblioteques, les eines
d'interacció, els jocs de proves i el simulador. Anomenam biblioteca a tot el
codi que l'usuari final no usa directament. Aquesta secció és el major cos de
la pràctica en nombre de línies perquè és la implementació del sistema de
fitxers pròpiament dit.

Les altres tres seccions formen el suport per l'ús del sistema de
fitxers. Anomenam a les eines d'interacció a tots els petits programes clàssics
del UNIX que s'usen al dia a dia: \verb+cd+, \verb+cat+, \verb+ls+ o
\verb+mkdir+. El simulador és la peça indicada a l'enunciat que fa la prova del
sistema. El simulador està completament definit i no és més que crear un gran
nombre de clients que facin un directori i duguin a terme escriptures a un
fitxer. Els jocs de proves són petits programes que tenen com a únic objectiu
comprovar la correcció de cadascun dels mòduls.

L'estructura de les biblioteques és un conjunt de vàries capes més un fitxer
comú a totes elles. L'estructura completa es pot estudiar a la taula
\ref{capes:simple} però bàsicament està formada pel nucli del sistema de
fitxers, les llibreries d'accés al nucli, la gestió de fitxers i les eines
d'usuari. Aquest element al que totes tenen accés s'anomena \verb+common.h+ i
allà s'emmagatzemen variables globals de configuració com el grandària de bloc
predeterminat. Les dades i valors que apareixen al fitxer comú poden
modificar-se a la creació de la imatge del SF.

\begin{table}
\centering
\begin{tabular}{|ll|}
\textbf{Nivell} & \textbf{Components} \\
\hline
simulació   & sim.c \\
\hline
eines & mi\_mkfs.c \\
      & mi\_rm.c \\
      & mi\_ln.c \\
      & $\cdots$ \\
\hline
fitxers & dir \\
        & file \\
        & inode \\
\hline
llibreries & block\_lib \\
\hline
nucli & block \\
      & super \\
      & bitmap \\
\hline
\end{tabular}
\caption{Esquema dels mòduls del SF}
\label{capes:simple}
\end{table}

\begin{table}
\centering
\begin{tabular}{|l|l|l|l|}
\hline
\multicolumn{3}{|c|}{Metadades} & Dades \\
\hline
Superbloc & Mapa de bits & Inodes & Dades \\
\hline
\end{tabular}
\caption{Esquema de l'ordenació de les dades del SF}
\label{esquema}
\end{table}


La informació permanent del sistema de fitxers està separada en dues parts: les
\emph{dades} i les seves \emph{metadades}. Les metadades estan compostes pel
\emph{superbloc}, el \emph{mapa de bits} i pel vector d'\emph{inodes}.  Aquest
esquema es pot estudiar a la taula \ref{esquema}. El superbloc s'encarrega de
mantenir informació com quants de blocs queden lliures o quin és el primer
inode i el mapa de bits té com a funció indicar quins blocs de dades estan
lliures i quins estan ocupats.

El vector d'inodes és un conjunt d'estructures que assenyalen les
característiques d'un fitxer o directori: el seu tamany, la darrera
modificació, quants d'enllaços té i on es poden trobar els blocs físics on es
desa la informació.  Un inode es pot comportar com un fitxer de dades, on es
desa informació, o com una entrada de directori, un text que sosté l'estructura
d'arbre dels fitxers.
