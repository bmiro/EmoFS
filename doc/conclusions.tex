\section{Conclusions}

Hem escrit des de zero un sistema de fitxers d'usuari amb inodes i directoris
en arbre. Hem programat les eines mínimes per treballar sobre el nostre SF i
executat una completa simulació per observar el seu comportament. Com a
resultat tenim un sistema de fitxers que malgrat no l'usariem al nostre dia a
dia té les característiques d'un sistema de fitxers clàssic. És una prova
conceptual completa del funcionament d'una peça important dels nostres sistemes
operatius.

D'aquesta manera tancam una pràctica on queda constància de la necessitat d'un
bon disseny, d'unes bones interfícies entre capes i de les infinites
possibilitats i característiques que li podem afegir. Algunes coses com el
\verb+FUSE+ \footnote{ \url{http://fuse.sourceforge.net/} }, un sistema de
fitxers a nivell d'usuari, o les memòries cau plantejaven un repte molt
interessant però la seva complexitat sobrepassa les expectatives de la
pràctica. De tota manera queda clar podriem dedicar-li moltíssimes més hores,
fer grups de quatre o sis persones i encara faltaria temps per implementar
totes les parts que defineix el que és un sistema de fitxers modern.
