
\section{Implementació}

Per simplicitat la gestió de concurrència es fa mitjançant un sol \emph{mutex}
que bloca totes les estructures. Aquest \emph{biglock} implica un rendiment
molt baix doncs sols permet una operació alhora però redueix enormement la
complexitat i els possibles problemes d'accés simultani. Tots els controls es
troben dins les funcions de \verb+dir.h+. També creàrem unes eines de
\verb+mi_mount+ i \verb+mi_umount+ que s'encarreguen de crear i destruir els
semàfors. Aquestes eines s'han d'executar abans de començar la interacció amb
el sistema de fitxers però no cal fer-ho pel simulador perquè el propi
\verb+sim.c+ ja ho fa.

Les variables i característiques com els tamanys de bloc, el nombre d'entrades
d'inode o el nombre de blocs de dades estan definits als fitxers corresponents:
\verb+common.h+, \verb+super.h+, \verb+inode.h+ \ldots Podem trobar que totes
les dades indirectes, com el tamany de \emph{padding} del superbloc o el nombre
final d'entrades al mapa de bits són calculades mitjançant macros però
recordant que s'ha recompilar el projecte si es vol fer algun canvi.

Com a funcions i tasques extres hem desenvolupat els programes d'usuari
\verb+mkdir+, \verb+touch+, \verb+append+. Amb aquests programes són molt útils
per fer comprovacions de com està la imatge del sistema de fitxers. Alhora així
podem escriure el simulador d'una manera especial: fent que cada tasca del fils
usi les eines d'usuari normals, imitant més el comportament d'usuari.

El \verb+ls+ accepta varis camins com entrada. És l'únic programa que ho fa
però sols ho hem volgut implementar com a prova conceptual: fer-ho per eines
com \verb+rm+ sols era un poc més de feina. Aquest processament es podria haver
fet amb una biblioteca d'arguments però l'hem fet a mà.

El programa \verb+mi_ls+ usa com a sortida un \verb+emofs_extract_path+ (el que
a l'enunciat és un \verb+info_fichero+) i aquestes funcions demanen un
\emph{buffer} per recórrer els fitxers d'un directori. Aquest \emph{buffer} pot
ser fixo, que malbarata l'espai, o dinàmic, amb el perill de quedar-se sense
memòria. Nosaltres hem usat la memòria dinàmica per així recordar els usos de
\verb+malloc+ i \verb+realloc+ sempre recordant que la pràctica no pretén tenir
directoris amb un gran nombre d'entrades.
