
\section{Descripció del problema}

El temari de l'assignatura d'ampliació de sistemes operatius és principalment
l'estudi de la memòria virtual i els sistemes de fitxers. Un sistema de fitxers
és la part del sistema operatiu encarregat de mantenir permanentment les dades
de l'usuari i els programes del sistema. Com a pràctica de l'assignatura
s'implementa un petit sistema de fitxers basat en inodes i amb capacitat de
directoris en arbre. Amb el llenguatge de programació C haurem d'escriure no
sols les eines típiques del UNIX com crear un directori o esborrar un fitxer
sinó també un simulador que executi una càrrega de treball contra el nostre
sistema.

Tal i per seguir les tradicions del UNIX vàrem decidir anomenar el nostre
sistema de fitxers (\emph{SF} a partir d'ara) algunacosa-fs. Com que un dels
membres del grup estava especialment trist en l'època de començar la pràctica
usarem \emph{emo} dels \emph{emotive hardcore}, gent que segons la cultura
popular sempre està deprimida i en conseqüència anomenam al nostre SF
\emph{emofs}.

L'estructura general de la pràctica ve indicada per l'enunciat exposat pels
professors: nou setmanes de feina on seqüencialment es van introduint distintes
capes de la feina. Des de la primera setmana on cream una imatge i escrivim
blocs d'informació fins a la darrera on treballam amb fitxers dins la nostra
imatge de sistema de fitxers.
